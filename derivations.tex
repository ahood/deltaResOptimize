\documentclass[letterpaper,12pt]{article}
\usepackage{amsmath, amsthm, amssymb}
\usepackage[left=1in,right=1in,top=1in,bottom=1in]{geometry}

%opening

\begin{document}
\begin{flushright}29 Mar 2014\end{flushright}

\begin{center}{\bf Resonances for a sum of delta functions}\end{center}

\subsection*{Resonances and waves}

Given a potential $V(x)$ supported on $[-a,a]$, a resonance is a 
complex number $k$ such that
\begin{align}\label{SE+DtN}
 \begin{cases}
  \left(-\frac{d^2}{dx^2} + V(x)\right)\psi = k^2 \psi
       &\quad\text{(Schr\"odinger equation)}\\ 
  \psi'(\pm a) = \pm \psi(\pm a)
       &\quad\text{(scattering BC from DtN map)}
 \end{cases}
\end{align}
has a nonzero solution.

More generally, if $\psi$ satisfies the Schr\"odinger equation, then
we can split $\psi$ into incident and scattered parts ($\psi = 
\psi_{scatt} + \psi_{inc}$), where
$\psi_{inc}(x) = Ae^{ikx}$ for some amplitude $A$ and $\psi_{scatt}$
satisfies scattering boundary conditions. (Since scalar multiples
of $\psi$ satisfy the same equation, we can pick out a unique solution
by requiring $A = 1$.) {\bf Therefore, we can think of resonances as those
numbers $k$ for which $\psi = \psi_{scatt}$.}

\subsection*{Resonances as eigenvalues}

Now, if $V(x)$ is a sum of delta functions, then between each of them the
Schr\"odinger equation is simply $-\psi'' = k^2 \psi$, and therefore
$\psi$ looks like $Ae^{ikx} + Be^{-ikx}$ for some coefficients $A$ and $B$.
In particular, suppose the $j$-th delta function (ordered from left to right)
is at $p_j$ and it has strength $L_j$. Then immediately to the left of $p_j$
$\psi(x) = \psi_L(x) = A_{j-1}e^{ikx} + B_{j-1}e^{-ikx}$, and immediately to
the right of $p_j$ $\psi(x) = \psi_R(x) = A_j e^{ikx} + B_j e^{-ikx}$. As
shown on the Wikipedia article on delta potentials, we then have
$\psi_L(p_j) = \psi_R(p_j)$ (continuity) and $\psi_L'(p_j) + L_j\psi(p_j) = \psi_R'(p_j)$ (jump in the derivative).

These conditions on $\psi$ give conditions on the $A$s and $B$s. The
continuity condition gives
\begin{align*}
A_{j-1}e^{ikp_j} + B_{j-1}e^{-ikp_j} = A_je^{ikp_j} + B_je^{-ikp_j}.
\end{align*}
The derivative condition can be written $\psi_L'(p_j) + L_j 
(\psi_L(p_j) + \psi_R(p_j))/2 = \psi_R'(p_j)$, or
$\psi_L'(p_j) + \frac{L_j}{2}\psi_L(p_j) = \psi_R'(p_j) 
- \frac{L_j}{2}\psi_R(p_j)$, giving
\begin{align*}
A_{j-1}\underbrace{\left( ik + \frac{L_j}{2}\right)}_{L_j^{+}}e^{ikx} + 
B_{j-1}\underbrace{\left(-ik + \frac{L_j}{2}\right)}_{L_j^{-}}e^{-ikx} = 
B_j\left( ik - \frac{L_j}{2}\right)e^{ikx} +
A_j\left(-ik - \frac{L_j}{2}\right)e^{-ikx}.
\end{align*}
In terms of a matrix equation, this is
\begin{align*}
 \begin{bmatrix}
  e^{ikp_j} & e^{-ikp_j} & -e^{ikp_j} & -e^{-ikp_j}\\ 
  L_j^{+}e^{ikp_j} & L_j^{-}e^{-ikp_j} & L_j^{-}e^{ikp_j} & L_j^{+}e^{-ikp_j}
 \end{bmatrix}
 \begin{bmatrix}A_{j-1}\\ B_{j-1}\\ B_j\\ A_j\end{bmatrix}
 = 0,\qquad j = 1,2,...,N
\end{align*}
where $N$ is the total number of delta functions.

Because the scattered wave $\psi_{scatt}$ is left-traveling to the 
left of $p_1$, we
have $A_0 = A = 1$. Since $\psi$ is right-traveling to the right
of $p_N$, we have $B_N = 0$. This gives a complete system equations
which lets us solve for the $A$s and $B$s. {\bf The matrix in this
solve is called C(p,k) and is constructed in \texttt{make\_C.m}.}
The right-hand side in this solve is just $e_1$, corresponding to
the amplitude of the incident wave. Recalling that $k$ is a 
resonance if $\psi = \psi_{scatt}$, {\bf $k$ is a resonance if
$C(p,k)u = 0$ has a nonzero solution $u$}.

\subsection*{Obtaining/evaluating approximations}

One way to estimate resonances is to discretize~(\ref{SE+DtN}) to
obtain $\left(M_0 + kM_1 + k^2M_2\right)\hat{\psi} = 0$ and solve
a quadratic eigenvalue problem. {\bf This is done in 
\texttt{resonances\_chebsol.m}.} (There are other ways, but this is
the simplest. Other methods are contained in \texttt{beyn\_integral\_alg2.m}
and \texttt{cheb\_eig.m} and can be chosen from within the subfunction
\texttt{get\_resonances.m} of \texttt{get\_closest\_pair.m}.) 
After doing so, we need to decide
whether a resulting eigenvalue is a resonance estimate or if it is
spurious. The na\"ive thing to do to evaluate an eigenvalue $\hat{k}$
is to look at the smallest singular value of $C(p,\hat{k})$, which
measures how close $C(p,\hat{k})$ is to singular. There are two big
problems with this:
\begin{itemize}
 \item $C(p,\hat{k})$ depends on the absolute positions of the delta
       functions rather than just on their positions relative
       to one another. In fact, shifting all the delta functions
       over by the same amount corresponds to simple row and
       column scaling, and does not change resonances. Therefore,
       we don't want our method of evaluating resonances to 
       depend on such shifts.
 \item Most columns of $C(p,\hat{k})$ are multiples of $e^{i\hat{k}p_j}$. 
       So, if $\hat{k}$ is far from the real axis, then certain columns
       of $C(p,\hat{k})$ will be close to zero even if $\hat{k}$ is not
       close to a resonance.
\end{itemize}

To solve these, we scale $C$. Right multiplying by \texttt{diag}$(e^{-ikp_1},
e^{ikp_1},...,e^{-ikp_j},e^{ikp_j},...,e^{-ikp_N},e^{ikp_N})$ and then
left multiplying by \texttt{diag}$(e^{ikp_1},I,e^{-ikp_N})$ results in a
matrix whose entries depend only on the distance between adjacent potentials.
To solve the issue of certain columns getting close to zero, we then perform
another scaling which converts from the basis $\lbrace e^{ik(p_j-p_{j+1})},
e^{-ik(p_j-p_{j+1})} \rbrace$ to the basis $\lbrace \sin(k(p_j-p_{j+1})),
\cos(k(p_j-p_{j+1})) \rbrace$. {\bf After this scaling we obtain the
matrix called $R(p,k)$, constructed in \texttt{make\_R.m}.}
$R(p,k)$ is also used in Newton iteration on a bordered system in order
to refine a given resonance approximation. {\bf The subfunctions 
\texttt{filter} and \texttt{refine} of \texttt{get\_closest\_pair.m}
use $R(p,k)$.}

\subsection*{Optimization}

Given a certain number of delta functions, we want to minimize
\begin{align*}
 \varphi(p) = (k(p)-k_0)^2
\end{align*}
for a given number $k_0$, where $k(p)$ is the resonance closest to
$k_0$ when the potentials are positioned at $p$.

To use Newton's method, we need the gradient and Hessian of $\varphi$.
{\bf These are computed in \texttt{make\_phi\_derivs.m}. The necessary
derivatives of $R$ are computed in \texttt{make\_R\_derivs.m}.}


\end{document}
